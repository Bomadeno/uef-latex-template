%Fill these in and they'll propagate across the title page. Remember to keep the space at the end!
\def \ajankohta {Tammikuu 2014 }
\def \ajankohtaenglish {January 2014 }
\def \authorname {John Smith }
\def \thesistitle {A Simple UEF LaTeX Template }
\def \thesissubtext { For Prettier Theses }
\def \campus {Joensuu }
\def \facultyschooleng {School of Computer Science }
\def \facultyschoolfin {Tietojenkäsittelytiede }
%FT = PhD, FM = MSc
\def \supervisorsfin {FT Jane Doe }
\def \supervisorseng {PhD Jane Doe }

\def \documenttypeeng {Master's Thesis }
\def \documenttypefin {Pro gradu -tutkielman }
%\def \documenttypeeng {Bachelor's Thesis}
%\def \thesistypefin {Kandidaatintutkielma}

%Number of non-cover pages, to last page of references
\def \mypagecount {42 }
\def \myappendixcount {1 }
\def \myappendixpagecount {32 }

%%%%%%%%%%%%%%%%%% TITLE PAGE %%%%%%%%%%%%%%%%%%

\vspace*{3cm}
\vspace{0.5cm}

\begin{center}
\begin{LARGE}\thesistitle \end{LARGE}

\begin{Large}\thesissubtext \end{Large} 

\vspace{1.5cm}

\begin{Large}\authorname \end{Large}

\vspace{\stretch{1}}

{\large
\documenttypeeng
~\\
% to have it in black and white, swap the commented line
\includegraphics[width=7cm]{UEF_fin_pysty_1_cmyk}\\
% \includegraphics[width=7cm]{UEF_fin_pysty_1_black}\\
Faculty of Science and Forestry\\
\facultyschooleng \\
\ajankohtaenglish \\
}
\end{center}

\vspace{0.5cm}

\thispagestyle{empty}


%%%%%%%%%%%%%%%%%% Abstract page Finnish %%%%%%%%%%%%%%%%%%
\begin{spacing}{1.0}
\begin{otherlanguage}{finnish}
\newpage


\pagenumbering{roman} % roman numeral page numbers, auto reset to 1

ITÄ-SUOMEN YLIOPISTO, Luonnontieteiden ja metsätieteiden tiedekunta, \campus Tietojenkäsittelytieteen laitos\\
\facultyschoolfin \\
\\ 
Opiskelija, \authorname : \thesistitle \\
\documenttypefin , \mypagecount s., \myappendixcount liite (\myappendixpagecount s.)\\
\documenttypefin ohjaajat: \supervisorsfin \\
\ajankohta \\


Tiivistelmä:

150-200 word abstract in Finnish, fit onto one page


% Key words in Finnish

% 5 kappaletta tutkielman sisältöä kuvaavia avainsanoja

% Seuraavassa olevat avainsanat kuvaavat tätä ohjetta; älä laita
% yhtäkään näistä omaan tutkielmaasi!

~\\ % Tämä tekee tyhjän rivin; älä editoi tätä pois
Avainsanat:
tutkielmaohje, tutkielman kirjoittaminen, tutkielman rakenne,
kandidaatintutkielma, pro gradu

% CR-luokat

% ACM-luokitus löytyy Computing Reviews -lehden jokaisen
% vuosikerran ensimmäisestä numerosta sekä verkosta
% osoitteesta http://www.acm.org/class/

% Ota omat luokkasi tuoreimmasta vuosikerrasta.

ACM-luokat (ACM Computing Classification System,
1998 version): A.m, K.3.2\\

\end{otherlanguage}
\newpage

%%%%%%%%%%%%%%%%%% Abstract page in English %%%%%%%%%%%%%%%%%%

UNIVERSITY OF EASTERN FINLAND, Faculty of Science and Forestry, \campus School of Computing\\
\facultyschooleng \\ \\
Student, \authorname : \thesistitle \\
\documenttypeeng , \mypagecount p., \myappendixcount appendix (\myappendixpagecount p.)\\
Supervisors of the \documenttypeeng : \supervisorseng \\
\ajankohtaenglish \\


Abstract:

150-200 word abstract in English, fit onto one page


% Key words English

% Edellisellä sivulla olevien suomenkielisten avainsanojen käännökset

~\\ % Tämä tekee tyhjän rivin; älä editoi tätä pois
Keywords:
keywords in english

% CR-luokat

% ACM-luokitus löytyy Computing Reviews -lehden jokaisen
% vuosikerran ensimmäisestä numerosta sekä verkosta
% osoitteesta http://www.acm.org/class/

% Ota omat luokkasi tuoreimmasta vuosikerrasta.

CR Categories (ACM Computing Classification System,
1998 version): A.m, K.3.2\\

\end{spacing}

\newpage


%%%%%%%%%%%%%%%%%% Foreword/Preface %%%%%%%%%%%%%%%%%%

\section*{Foreword}
Up to one page forweword, for thanks and admirations.

\newpage

%%%%%%%%%%%%%%%%%% Abbrieviations %%%%%%%%%%%%%%%%%%

\section*{List of Abbreviations}

\begin{tabular}{lp{12.5cm}}

ACM & Association for Computing Machinery \\

ISY & Itä-Suomen yliopisto \\

UEF & University of Eastern Finland\\

\end{tabular}

\newpage


% ----------------- Table of Contents -------------

% Älä tee seuraavaan mitään muutoksia:

\setlength{\parskip}{0ex}

\tableofcontents
\newpage

\setlength{\parskip}{2ex}
